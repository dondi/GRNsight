\documentclass[sigconf,review,anonymous]{acmart}

\usepackage{booktabs} % For formal tables

% Copyright
\setcopyright{none}
%\setcopyright{acmcopyright}
%\setcopyright{acmlicensed}
%\setcopyright{rightsretained}
%\setcopyright{usgov}
%\setcopyright{usgovmixed}
%\setcopyright{cagov}
%\setcopyright{cagovmixed}


% DOI
\acmDOI{10.475/123_4}

% ISBN
\acmISBN{123-4567-24-567/17/06}

%Conference
\acmConference[SIGGRAPH 2017 Posters]{SIGGRAPH 2017 Posters}{August 2017}{Los Angeles, CA, USA} 
\acmYear{2017}
\copyrightyear{2017}
\acmPrice{15.00}

% use the "authoryear" citation style.
\citestyle{acmauthoryear}
\setcitestyle{square}

\begin{document}
\title{GRNsight: a web application and service for visualizing models of small- to medium-scale gene regulatory networks}

\author{Eileen Choe}
\orcid{1234-5678-9012}
\affiliation{%
  \institution{Loyola Marymount University}
  \streetaddress{1 LMU Drive}
  \city{Los Angeles} 
  \state{California} 
  \postcode{43017-6221}
}
\email{trovato@corporation.com}

\author{G.K.M. Tobin}
\affiliation{%
  \institution{Institute for Clarity in Documentation}
  \streetaddress{P.O. Box 1212}
  \city{Dublin} 
  \state{Ohio} 
  \postcode{43017-6221}
}
\email{webmaster@marysville-ohio.com}

% The default list of authors is too long for headers}
\renewcommand{\shortauthors}{B. Trovato et. al.}

\begin{abstract}

GRNsight is a web application and service for visualizing small- to medium-scale gene regulatory networks (GRNs). A GRN consists of genes, transcription factors, and the regulatory connections between them which govern the level of expression of mRNA and protein from genes. GRNsight produces weighted or unweighted network graphs from an Excel spreadsheet containing an adjacency matrix where regulators are named in the columns and target genes in the rows, a Simple Interaction Format (SIF) text file, or a GraphML XML file. GRNsight represents genes as nodes and regulatory connections as edges with colors, end markers, and thicknesses corresponding to the sign and magnitude of activation or repression. GRNsight visualizations can be modified through manually dragging nodes or adjusting sliders that change the force graph parameters. GRNsight is best-suited for visualizing networks of fewer than 35 nodes and 70 edges, although it accepts networks of up to 75 nodes or 150 edges. GRNsight has general applicability for displaying any small, unweighted or weighted network with directed edges for systems biology or other application domains. The GRNsight application (http://dondi.github.io/GRNsight/) and code (https://github.com/dondi/GRNsight) are available under the open source BSD license.

\end{abstract}

%
% The code below should be generated by the tool at
% http://dl.acm.org/ccs.cfm
% Please copy and paste the code instead of the example below. 
%
\begin{CCSXML}
<ccs2012>
<concept>
<concept_id>10003120.10003145.10003147.10010364</concept_id>
<concept_desc>Human-centered computing~Scientific visualization</concept_desc>
<concept_significance>500</concept_significance>
</concept>
<concept>
<concept_id>10003120.10003145.10003151.10011771</concept_id>
<concept_desc>Human-centered computing~Visualization toolkits</concept_desc>
<concept_significance>500</concept_significance>
</concept>
</ccs2012>
\end{CCSXML}

\ccsdesc[500]{Human-centered computing~Scientific visualization}
\ccsdesc[500]{Human-centered computing~Visualization toolkits}

% We no longer use \terms command
%\terms{Theory}

\keywords{ACM proceedings, \LaTeX, text tagging}

\begin{teaserfigure}
  %\includegraphics[width=\textwidth]{sampleteaser}
  \caption{This is a teaser image.}
  \label{fig:teaser}
\end{teaserfigure}


\maketitle

\section{Introduction and Motivation}

GRNsight is a web application and service for visualizing models of small- to medium-scale gene regulatory networks (GRNs). A gene regulatory network (GRN) consists of genes, transcription factors, and the regulatory connections between them which govern the level of expression of mRNA and protein from genes. The original motivation came from our efforts to perform parameter estimation and forward simulation of the dynamics of a differential equations model of a small GRN with 21 nodes and 31 edges, and quickly and easily visualize the weight parameters from the model. A review by Pavlopoulos et al. \cite{doi:10.1186/s13742-015-0077-2}, describes the types, trends, and usage of visualization tools available for genomics and systems biology. Their list of 47 tools for network analysis is representative of what was available to us at our project inception in January 2014. However, our use case was narrow, because we wanted a tool for novice and experienced biologists alike to quickly and easily view the unweighted and weighted network graphs corresponding to the matrix without having to create or modify MATLAB code, and the tools we investigated out of this diverse set each had properties that limited their use for us. Thus, we enumerated the following requirements for a potential visualization tool. The tool should:

\begin{enumerate}
\item Exist as a web application without the need to download and install specialized software;
\item Be simple and intuitive to use;
\item Automatically lay out and display small- to medium-scale, unweighted and weighted, directed network graphs in a way that is familiar to biologists and adds value to the interpretation of the modeling results.
\end{enumerate}

\begin{figure}[ht]
  \centering
  %\includegraphics[width=3.0in]{ferrari_laferrari}
  \caption{Ferrari LaFerrari. (Image courtesy Flickr user ``gfreeman23.'')}
  \label{fig:ferrari}
\end{figure}

Maecenas pharetra libero ac sapien accumsan, iaculis suscipit ex fringilla. Etiam urna mauris, maximus at sapien sed, semper hendrerit libero. Lorem ipsum dolor sit amet, consectetur adipiscing elit. Suspendisse potenti. Phasellus felis velit, finibus at felis a, commodo mollis odio. Praesent efficitur lobortis quam, et volutpat erat dignissim eu. 

\section{Materials and Methods}

\subsection{Graph Customizations}
GRNsight's diagrams are based on force graph layout algorithms in the D3.js visualization library \cite{d3}, which was then extensively customized to support the specific needs of biologists for GRN visualization. D3.js's baseline force graph implementation had round, unlabeled nodes and undirected, straight-line edges. The following customizations were made for the nodes: (a) the nodes were made rectangular; (b) a label of up to 12 characters was added; (c) node size was varied, depending on the size of the label.

\subsection{User Interface}



\begin{figure}[h]
%\includegraphics[height=1in, width=1in]{rosette}
\caption{A sample black and white graphic that has
been resized with the \texttt{includegraphics} command.}
\end{figure}


\section{Results and Discussion}


\subsection{A Subsection}


\subsection{Another Subsection}

\section{Conclusion and Future Work}


\bibliographystyle{ACM-Reference-Format}
\bibliography{siggraph-abstract-review} 

\end{document}
