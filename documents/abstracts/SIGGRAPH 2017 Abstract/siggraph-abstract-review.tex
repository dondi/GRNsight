\documentclass[sigconf,review,anonymous]{acmart}

\usepackage{booktabs} % For formal tables

% Copyright
\setcopyright{none}
%\setcopyright{acmcopyright}
%\setcopyright{acmlicensed}
%\setcopyright{rightsretained}
%\setcopyright{usgov}
%\setcopyright{usgovmixed}
%\setcopyright{cagov}
%\setcopyright{cagovmixed}


% DOI
\acmDOI{10.475/123_4}

% ISBN
\acmISBN{123-4567-24-567/17/06}

%Conference
\acmConference[SIGGRAPH 2017 Posters]{SIGGRAPH 2017 Posters}{August 2017}{Los Angeles, CA, USA} 
\acmYear{2017}
\copyrightyear{2017}
\acmPrice{15.00}

% use the "authoryear" citation style.
\citestyle{acmauthoryear}
\setcitestyle{square}

\begin{document}
\title{GRNsight: a web application and service for visualizing models of small- to medium-scale gene regulatory networks}

\author{Eileen Choe}
\orcid{1234-5678-9012}
\affiliation{%
  \institution{Loyola Marymount University}
  \streetaddress{1 LMU Drive}
  \city{Los Angeles} 
  \state{California} 
  \postcode{43017-6221}
}
\email{trovato@corporation.com}

\author{G.K.M. Tobin}
\affiliation{%
  \institution{Institute for Clarity in Documentation}
  \streetaddress{P.O. Box 1212}
  \city{Dublin} 
  \state{Ohio} 
  \postcode{43017-6221}
}
\email{webmaster@marysville-ohio.com}

% The default list of authors is too long for headers}
\renewcommand{\shortauthors}{B. Trovato et. al.}

\begin{abstract}

GRNsight is a web application and service for visualizing small- to medium-scale gene regulatory networks (GRNs). A GRN consists of genes, transcription factors, and the regulatory connections between them which govern the level of expression of mRNA and protein from genes. GRNsight produces weighted or unweighted network graphs from an Excel spreadsheet containing an adjacency matrix where regulators are named in the columns and target genes in the rows, a Simple Interaction Format (SIF) text file, or a GraphML XML file. GRNsight represents genes as nodes and regulatory connections as edges with colors, end markers, and thicknesses corresponding to the sign and magnitude of activation or repression. GRNsight visualizations can be modified through manually dragging nodes or adjusting sliders that change the force graph parameters. GRNsight is best-suited for visualizing networks of fewer than 35 nodes and 70 edges, although it accepts networks of up to 75 nodes or 150 edges. GRNsight has general applicability for displaying any small, unweighted or weighted network with directed edges for systems biology or other application domains. The GRNsight application (http://dondi.github.io/GRNsight/) and code (https://github.com/dondi/GRNsight) are available under the open source BSD license.

\end{abstract}

%
% The code below should be generated by the tool at
% http://dl.acm.org/ccs.cfm
% Please copy and paste the code instead of the example below. 
%
\begin{CCSXML}
<ccs2012>
<concept>
<concept_id>10003120.10003145.10003147.10010364</concept_id>
<concept_desc>Human-centered computing~Scientific visualization</concept_desc>
<concept_significance>500</concept_significance>
</concept>
<concept>
<concept_id>10003120.10003145.10003151.10011771</concept_id>
<concept_desc>Human-centered computing~Visualization toolkits</concept_desc>
<concept_significance>500</concept_significance>
</concept>
</ccs2012>
\end{CCSXML}

\ccsdesc[500]{Human-centered computing~Scientific visualization}
\ccsdesc[500]{Human-centered computing~Visualization toolkits}

% We no longer use \terms command
%\terms{Theory}

\keywords{ACM proceedings, \LaTeX, text tagging}

\begin{teaserfigure}
  %\includegraphics[width=\textwidth]{sampleteaser}
  \caption{This is a teaser image.}
  \label{fig:teaser}
\end{teaserfigure}


\maketitle

\section{Introduction and Motivation}

GRNsight is a web application and service for visualizing models of small- to medium-scale gene regulatory networks (GRNs). A gene regulatory network (GRN) consists of genes, transcription factors, and the regulatory connections between them which govern the level of expression of mRNA and protein from genes. A review by Pavlopoulos et al. \cite{doi:10.1186/s13742-015-0077-2}, describes the types, trends, and usage of visualization tools available for genomics and systems biology. Their list of 47 tools for network analysis is representative of what was available to us at our project inception in January 2014 (given the caveat that the list itself is a moving target with some tools dropping out, new ones being added, and others evolving in their functions). With such a large number of tools available, it would be reasonable to expect that one already existed that could fulfill our needs. However, our use case was narrow, and the tools we investigated out of this diverse set each had properties that limited their use for us.

Requirements of our tool:
\begin{enumerate}
\item Exist as a web application without the need to download and install specialized software;
\item Be simple and intuitive to use;
\item Automatically lay out and display small- to medium-scale, unweighted and weighted, directed network graphs in a way that is familiar to biologists and adds value to the interpretation of the modeling results.
\end{enumerate}

\begin{figure}[ht]
  \centering
  %\includegraphics[width=3.0in]{ferrari_laferrari}
  \caption{Ferrari LaFerrari. (Image courtesy Flickr user ``gfreeman23.'')}
  \label{fig:ferrari}
\end{figure}

Maecenas pharetra libero ac sapien accumsan, iaculis suscipit ex fringilla. Etiam urna mauris, maximus at sapien sed, semper hendrerit libero. Lorem ipsum dolor sit amet, consectetur adipiscing elit. Suspendisse potenti. Phasellus felis velit, finibus at felis a, commodo mollis odio. Praesent efficitur lobortis quam, et volutpat erat dignissim eu. 

\section{Materials and Methods}

Cum sociis natoque penatibus et magnis dis parturient montes, nascetur ridiculus mus. Vivamus maximus a lectus sed dictum. Curabitur pulvinar lectus nec magna molestie consequat. Aliquam lacinia quam ac tristique sodales. Class aptent taciti sociosqu ad 
% Numbered Equation
\begin{equation}
\label{eqn:01}
P(t)=\frac{b^{\frac{t+1}{T+1}}-b^{\frac{t}{T+1}}}{b-1},
\end{equation}
where $t=0,{\ldots}\,,T$, and $b$ is a number greater than $1$, litora torquent per conubia nostra, per inceptos himenaeos.

Cras tempus libero nunc, ac suscipit mi varius rutrum. Sed non nisl felis. Nunc a cursus elit. Fusce quam enim, congue id malesuada vel, ullamcorper sit amet ipsum. Cras non lobortis eros, sit amet hendrerit dui. Aenean semper eros non eros ornare, vitae efficitur nunc consequat. 

\begin{multline}
\label{the-rendering-equation}
L_o(x, \omega_o, \lambda, t) = L_e(x, \omega_o, \lambda, t)  + \\
\int_{\Omega} f_r(x, \omega_i, \omega_o, \lambda, t) L_i(x, \omega_i, \lambda, t)(\omega_i \cdot n) \text{d} \omega_i
\end{multline}

(Yes, that's the Rendering Equation.)~\cite{Kajiya:1986:RE:15922.15902}. Aenean pharetra ipsum eu mi fermentum dictum. Maecenas vel dolor semper, efficitur elit eget, bibendum diam. Duis vitae varius nisl. Proin aliquet sapien enim, eu vehicula ipsum euismod ut. Curabitur quis luctus quam, at ultricies ligula. Etiam imperdiet efficitur ipsum eu feugiat. Maecenas quis laoreet eros. Morbi molestie ac dui ac vestibulum. Donec maximus ex at neque posuere, et blandit tellus iaculis.

\begin{figure}[h]
%\includegraphics[height=1in, width=1in]{rosette}
\caption{A sample black and white graphic that has
been resized with the \texttt{includegraphics} command.}
\end{figure}

Aliquam erat volutpat. Vestibulum vestibulum dictum dui. Vestibulum ultricies turpis augue. Phasellus nec lacus malesuada, gravida felis vitae, tristique elit. In id sagittis arcu. Etiam euismod ex sit amet hendrerit volutpat. Vestibulum vel molestie magna. Suspendisse in tellus et mi tincidunt bibendum a at dui. Curabitur ac arcu tincidunt, mattis dui ut, commodo metus.

\section{Results and Discussion}

Aenean vestibulum sapien eget nulla volutpat elementum. Nam porttitor egestas felis ac commodo. Maecenas eleifend nisi in ligula accumsan, et pretium metus congue. In elementum ligula eget mi rhoncus gravida. Pellentesque est nunc, hendrerit at sapien sed, egestas sollicitudin risus. Aliquam erat volutpat. Integer at enim quam. Phasellus vitae ex non neque rutrum ornare. Aliquam bibendum magna ut tincidunt tincidunt. Pellentesque habitant morbi tristique senectus et netus et malesuada fames ac turpis egestas.

\subsection{A Subsection}

Praesent porttitor venenatis leo, at fermentum diam vestibulum non. Donec eu ultricies urna. In dictum finibus lectus non condimentum. In et ipsum dapibus, tempor ligula vitae, aliquet nunc. Sed posuere ligula et metus viverra consequat. 
\begin{equation}
\begin{split}
F = \{F_{x} \in  F_{c} &: (|S| > |C|) \\
 &\quad \cap (\text{minPixels}  < |S| < \text{maxPixels}) \\
 &\quad \cap (|S_{\text{connected}}| > |S| - \epsilon) \}
\end{split}
\end{equation}
Sed vel erat eu purus gravida tristique at ac mi. Cras tincidunt tristique nisl eget fermentum. Nam sodales tempor felis non scelerisque. Donec vitae accumsan metus. Aliquam laoreet eget nibh at ullamcorper. Nam in mollis orci, et porta massa. Etiam non odio a mi maximus ornare.

\subsection{Another Subsection}

Curabitur ac feugiat odio, ut molestie sem. Vestibulum ultricies tellus nibh, a faucibus justo feugiat accumsan. In cursus nibh elementum, posuere erat sed, egestas odio. Nulla volutpat lacinia ex, a aliquam neque bibendum vel. Integer efficitur, eros ut varius maximus, quam lacus dictum sapien, nec tempus neque neque feugiat metus. 

\section{Conclusion and Future Work}

Morbi sodales iaculis dolor id finibus. Cras bibendum odio nulla, eget sodales tortor posuere nec. Nulla eu massa odio. Phasellus fringilla massa nec augue maximus, vitae lacinia lectus sodales. Etiam nec placerat leo. Class aptent taciti sociosqu ad litora torquent per conubia nostra, per inceptos himenaeos. In hac habitasse platea dictumst. Sed at condimentum leo, id tincidunt velit. Etiam volutpat tempus aliquet. Praesent sollicitudin arcu et eleifend tincidunt. Quisque convallis, dui ac tristique auctor, dolor libero ultrices ligula, quis mollis diam diam in dolor. Nunc sed vehicula ligula, eget dignissim arcu.

\bibliographystyle{ACM-Reference-Format}
\bibliography{siggraph-abstract-review} 

\end{document}
